% Options for packages loaded elsewhere
\PassOptionsToPackage{unicode}{hyperref}
\PassOptionsToPackage{hyphens}{url}
\PassOptionsToPackage{dvipsnames,svgnames,x11names}{xcolor}
%
\documentclass[
]{article}

\usepackage{amsmath,amssymb}
\usepackage{iftex}
\ifPDFTeX
  \usepackage[T1]{fontenc}
  \usepackage[utf8]{inputenc}
  \usepackage{textcomp} % provide euro and other symbols
\else % if luatex or xetex
  \usepackage{unicode-math}
  \defaultfontfeatures{Scale=MatchLowercase}
  \defaultfontfeatures[\rmfamily]{Ligatures=TeX,Scale=1}
\fi
\usepackage{lmodern}
\ifPDFTeX\else  
    % xetex/luatex font selection
\fi
% Use upquote if available, for straight quotes in verbatim environments
\IfFileExists{upquote.sty}{\usepackage{upquote}}{}
\IfFileExists{microtype.sty}{% use microtype if available
  \usepackage[]{microtype}
  \UseMicrotypeSet[protrusion]{basicmath} % disable protrusion for tt fonts
}{}
\makeatletter
\@ifundefined{KOMAClassName}{% if non-KOMA class
  \IfFileExists{parskip.sty}{%
    \usepackage{parskip}
  }{% else
    \setlength{\parindent}{0pt}
    \setlength{\parskip}{6pt plus 2pt minus 1pt}}
}{% if KOMA class
  \KOMAoptions{parskip=half}}
\makeatother
\usepackage{xcolor}
\setlength{\emergencystretch}{3em} % prevent overfull lines
\setcounter{secnumdepth}{5}
% Make \paragraph and \subparagraph free-standing
\makeatletter
\ifx\paragraph\undefined\else
  \let\oldparagraph\paragraph
  \renewcommand{\paragraph}{
    \@ifstar
      \xxxParagraphStar
      \xxxParagraphNoStar
  }
  \newcommand{\xxxParagraphStar}[1]{\oldparagraph*{#1}\mbox{}}
  \newcommand{\xxxParagraphNoStar}[1]{\oldparagraph{#1}\mbox{}}
\fi
\ifx\subparagraph\undefined\else
  \let\oldsubparagraph\subparagraph
  \renewcommand{\subparagraph}{
    \@ifstar
      \xxxSubParagraphStar
      \xxxSubParagraphNoStar
  }
  \newcommand{\xxxSubParagraphStar}[1]{\oldsubparagraph*{#1}\mbox{}}
  \newcommand{\xxxSubParagraphNoStar}[1]{\oldsubparagraph{#1}\mbox{}}
\fi
\makeatother


\providecommand{\tightlist}{%
  \setlength{\itemsep}{0pt}\setlength{\parskip}{0pt}}\usepackage{longtable,booktabs,array}
\usepackage{calc} % for calculating minipage widths
% Correct order of tables after \paragraph or \subparagraph
\usepackage{etoolbox}
\makeatletter
\patchcmd\longtable{\par}{\if@noskipsec\mbox{}\fi\par}{}{}
\makeatother
% Allow footnotes in longtable head/foot
\IfFileExists{footnotehyper.sty}{\usepackage{footnotehyper}}{\usepackage{footnote}}
\makesavenoteenv{longtable}
\usepackage{graphicx}
\makeatletter
\newsavebox\pandoc@box
\newcommand*\pandocbounded[1]{% scales image to fit in text height/width
  \sbox\pandoc@box{#1}%
  \Gscale@div\@tempa{\textheight}{\dimexpr\ht\pandoc@box+\dp\pandoc@box\relax}%
  \Gscale@div\@tempb{\linewidth}{\wd\pandoc@box}%
  \ifdim\@tempb\p@<\@tempa\p@\let\@tempa\@tempb\fi% select the smaller of both
  \ifdim\@tempa\p@<\p@\scalebox{\@tempa}{\usebox\pandoc@box}%
  \else\usebox{\pandoc@box}%
  \fi%
}
% Set default figure placement to htbp
\def\fps@figure{htbp}
\makeatother

\makeatletter
\@ifpackageloaded{caption}{}{\usepackage{caption}}
\AtBeginDocument{%
\ifdefined\contentsname
  \renewcommand*\contentsname{Table of contents}
\else
  \newcommand\contentsname{Table of contents}
\fi
\ifdefined\listfigurename
  \renewcommand*\listfigurename{List of Figures}
\else
  \newcommand\listfigurename{List of Figures}
\fi
\ifdefined\listtablename
  \renewcommand*\listtablename{List of Tables}
\else
  \newcommand\listtablename{List of Tables}
\fi
\ifdefined\figurename
  \renewcommand*\figurename{Figure}
\else
  \newcommand\figurename{Figure}
\fi
\ifdefined\tablename
  \renewcommand*\tablename{Table}
\else
  \newcommand\tablename{Table}
\fi
}
\@ifpackageloaded{float}{}{\usepackage{float}}
\floatstyle{ruled}
\@ifundefined{c@chapter}{\newfloat{codelisting}{h}{lop}}{\newfloat{codelisting}{h}{lop}[chapter]}
\floatname{codelisting}{Listing}
\newcommand*\listoflistings{\listof{codelisting}{List of Listings}}
\makeatother
\makeatletter
\makeatother
\makeatletter
\@ifpackageloaded{caption}{}{\usepackage{caption}}
\@ifpackageloaded{subcaption}{}{\usepackage{subcaption}}
\makeatother

\usepackage{bookmark}

\IfFileExists{xurl.sty}{\usepackage{xurl}}{} % add URL line breaks if available
\urlstyle{same} % disable monospaced font for URLs
\hypersetup{
  pdftitle={過去問分析(Exam2024\_prb)},
  colorlinks=true,
  linkcolor={blue},
  filecolor={Maroon},
  citecolor={Blue},
  urlcolor={Blue},
  pdfcreator={LaTeX via pandoc}}


\title{過去問分析(Exam2024\_prb)}
\author{}
\date{}

\begin{document}
\maketitle

\renewcommand*\contentsname{Table of contents}
{
\hypersetup{linkcolor=}
\setcounter{tocdepth}{3}
\tableofcontents
}

\section{対象}\label{ux5bfeux8c61}

Exam2024\_prb(問題1〜6)

\begin{center}\rule{0.5\linewidth}{0.5pt}\end{center}

\section{問題1(空欄補充:微分・極値・重積分・変数変換)}\label{ux554fux984c1ux7a7aux6b04ux88dcux5145ux5faeux5206ux6975ux5024ux91cdux7a4dux5206ux5909ux6570ux5909ux63db}

\subsection{\texorpdfstring{(1)
2階偏導関数:\(z=\log|x^2-2y|\)}{(1) 2階偏導関数:z=\textbackslash log\textbar x\^{}2-2y\textbar{}}}\label{ux968eux504fux5c0eux95a2ux6570zlogx2-2y}

\subsubsection{解くための公式}\label{ux89e3ux304fux305fux3081ux306eux516cux5f0f}

置換 \(u=x^2-2y\) とすると {[} z=\log\textbar u\textbar,\qquad
\frac{\partial z}{\partial x}=\frac{1}{u}\frac{\partial u}{\partial x},\qquad
\frac{\partial z}{\partial y}=\frac{1}{u}\frac{\partial u}{\partial y}\quad(u\neq 0).
{]} 2階偏導・混合偏導は {[}
\frac{\partial^2 z}{\partial x^2}=\frac{\partial}{\partial x}\left(\frac{\partial z}{\partial x}\right),\qquad
\frac{\partial^2 z}{\partial y\,\partial x}=\frac{\partial}{\partial y}\left(\frac{\partial z}{\partial x}\right).
{]}

\subsubsection{使う分野の知識}\label{ux4f7fux3046ux5206ux91ceux306eux77e5ux8b58}

\begin{itemize}
\tightlist
\item
  多変数微分、合成関数(連鎖律)
\item
  \(\log|u|\) の定義域(\(u\neq 0\))
\end{itemize}

\subsubsection{解くための考え方}\label{ux89e3ux304fux305fux3081ux306eux8003ux3048ux65b9}

\begin{itemize}
\tightlist
\item
  まず \(u\) を置いて \(\partial z/\partial x,\partial z/\partial y\)
  を作る
\item
  その式をもう一回微分して2階へ(分数微分の符号ミスに注意)
\end{itemize}

\begin{center}\rule{0.5\linewidth}{0.5pt}\end{center}

\subsection{\texorpdfstring{(2)
連鎖律:\(z=f(x,y),\ x=a\cos t,\ y=b\sin t\)}{(2) 連鎖律:z=f(x,y),\textbackslash{} x=a\textbackslash cos t,\textbackslash{} y=b\textbackslash sin t}}\label{ux9023ux9396ux5f8bzfxy-xacos-t-ybsin-t}

\subsubsection{解くための公式}\label{ux89e3ux304fux305fux3081ux306eux516cux5f0f-1}

1階: {[} \frac{dz}{dt}=f\_x\frac{dx}{dt}+f\_y\frac{dy}{dt}. {]}
2階(標準形): {[} \frac{d^2z}{dt^2}
=f\_\{xx\}\left(\frac{dx}{dt}\right)\^{}2
+2f\_\{xy\}\left(\frac{dx}{dt}\right)\left(\frac{dy}{dt}\right)
+f\_\{yy\}\left(\frac{dy}{dt}\right)\^{}2 +f\_x\frac{d^2x}{dt^2}
+f\_y\frac{d^2y}{dt^2}. {]}

\subsubsection{使う分野の知識}\label{ux4f7fux3046ux5206ux91ceux306eux77e5ux8b58-1}

\begin{itemize}
\tightlist
\item
  多変数の連鎖律(1階・2階)
\item
  偏微分記号(\(f_x,f_{xy}\) など)
\end{itemize}

\subsubsection{解くための考え方}\label{ux89e3ux304fux305fux3081ux306eux8003ux3048ux65b9-1}

\begin{itemize}
\tightlist
\item
  先に \(dx/dt,dy/dt,d^2x/dt^2,d^2y/dt^2\) を確定(\(\sin,\cos\)
  の微分)
\item
  2階は「2階偏微分ブロック」+「1階偏微分×2階導関数ブロック」に分けて整理
\end{itemize}

\begin{center}\rule{0.5\linewidth}{0.5pt}\end{center}

\subsection{\texorpdfstring{(3) 2次関数の最大値:\(f,g,h\)
のうち最大値をもつもの}{(3) 2次関数の最大値:f,g,h のうち最大値をもつもの}}\label{ux6b21ux95a2ux6570ux306eux6700ux5927ux5024fgh-ux306eux3046ux3061ux6700ux5927ux5024ux3092ux3082ux3064ux3082ux306e}

\subsubsection{解くための公式}\label{ux89e3ux304fux305fux3081ux306eux516cux5f0f-2}

2変数2次関数 \(Q(x,y)\) に対し、ヘッセ行列 {[} H=

\begin{pmatrix}Q_{xx}&Q_{xy}\\Q_{yx}&Q_{yy}\end{pmatrix}

,\qquad D=\det H=Q\_\{xx\}Q\_\{yy\}-Q\_\{xy\}\^{}2. {]} 判定: {[}
D\textgreater0,~Q\_\{xx\}\textless0
\Rightarrow \text{極大(負定値)},\qquad
D\textgreater0,~Q\_\{xx\}\textgreater0
\Rightarrow \text{極小(正定値)},\qquad D\textless0
\Rightarrow \text{鞍点(極値なし)}. {]} 極値点は
\(\nabla Q=0\)(連立一次方程式)で求まる。2次関数が負定値なら全体最大値をもつ。

\subsubsection{使う分野の知識}\label{ux4f7fux3046ux5206ux91ceux306eux77e5ux8b58-2}

\begin{itemize}
\tightlist
\item
  多変数関数の極値、ヘッセ行列、二次形式
\end{itemize}

\subsubsection{解くための考え方}\label{ux89e3ux304fux305fux3081ux306eux8003ux3048ux65b9-2}

\begin{itemize}
\tightlist
\item
  まず \(H\) の符号(負定値か)で「最大値を持ちうるか」を絞る
\item
  候補だけ \(\nabla Q=0\) を解いて点を出し、代入して値を比較
\end{itemize}

\begin{center}\rule{0.5\linewidth}{0.5pt}\end{center}

\subsection{\texorpdfstring{(4) 二重積分:\(\iint_D x\,dx\,dy\)(\(y\)
の上下が2つの放物線)}{(4) 二重積分:\textbackslash iint\_D x\textbackslash,dx\textbackslash,dy(y の上下が2つの放物線)}}\label{ux4e8cux91cdux7a4dux5206iint_d-xdxdyy-ux306eux4e0aux4e0bux304c2ux3064ux306eux653eux7269ux7dda}

\subsubsection{解くための公式}\label{ux89e3ux304fux305fux3081ux306eux516cux5f0f-3}

縦に積分: {[} \iint\emph{\{D\} x,dx,dy = \int}\{\alpha\}\^{}\{\beta\}
\left( \int\emph{\{y=\ell(x)\}\^{}\{u(x)\} x,dy \right),dx. {]} 内側は
\(x\) が \(y\) に依らないので {[} \int}\{y=\ell(x)\}\^{}\{u(x)\} x,dy =
x\bigl(u(x)-\ell(x)\bigr). {]} ここで \(x\) の範囲 \([\alpha,\beta]\)
は、領域が存在する条件 \(\ell(x)\le u(x)\)(上下曲線の交点)から決める。

\subsubsection{使う分野の知識}\label{ux4f7fux3046ux5206ux91ceux306eux77e5ux8b58-3}

\begin{itemize}
\tightlist
\item
  二重積分、領域の切り出し、交点(2次方程式)
\end{itemize}

\subsubsection{解くための考え方}\label{ux89e3ux304fux305fux3081ux306eux8003ux3048ux65b9-3}

\begin{itemize}
\tightlist
\item
  交点で \(\alpha,\beta\) を確定
\item
  内側は「長さ \(u(x)-\ell(x)\) × \(x\)」に落として外側を1変数積分で処理
\end{itemize}

\begin{center}\rule{0.5\linewidth}{0.5pt}\end{center}

\subsection{\texorpdfstring{(5)
変数変換:\(0\le x+y\le 2,\ 0\le 3x-y\le 3\) の領域で
\(\iint_D y\,dx\,dy\)}{(5) 変数変換:0\textbackslash le x+y\textbackslash le 2,\textbackslash{} 0\textbackslash le 3x-y\textbackslash le 3 の領域で \textbackslash iint\_D y\textbackslash,dx\textbackslash,dy}}\label{ux5909ux6570ux5909ux63db0le-xyle-2-0le-3x-yle-3-ux306eux9818ux57dfux3067-iint_d-ydxdy}

\subsubsection{解くための公式}\label{ux89e3ux304fux305fux3081ux306eux516cux5f0f-4}

線形変換 {[} u=x+y,\qquad v=3x-y. {]} ヤコビアン: {[}
dx,dy=\left\textbar{}\frac{\partial(x,y)}{\partial(u,v)}\right\textbar,du,dv
=\frac{1}{\left|\frac{\partial(u,v)}{\partial(x,y)}\right|},du,dv. {]}
(\(\partial(u,v)/\partial(x,y)\) は一定なので計算は楽)
また、被積分関数 \(y\) を \(y(u,v)\) に直すために、\(u,v\) から \(x,y\)
を連立一次で解く。

\subsubsection{使う分野の知識}\label{ux4f7fux3046ux5206ux91ceux306eux77e5ux8b58-4}

\begin{itemize}
\tightlist
\item
  変数変換(2変数)、ヤコビアン、逆変換(連立一次)
\end{itemize}

\subsubsection{解くための考え方}\label{ux89e3ux304fux305fux3081ux306eux8003ux3048ux65b9-4}

\begin{itemize}
\tightlist
\item
  条件がそのまま \(0\le u\le 2,\ 0\le v\le 3\) の長方形になる(典型)
\item
  \(y(u,v)\) と \(|J|\) を出して長方形上の通常の二重積分にする
\end{itemize}

\begin{center}\rule{0.5\linewidth}{0.5pt}\end{center}

\section{問題2(媒介変数曲線:長さ・面積)}\label{ux554fux984c2ux5a92ux4ecbux5909ux6570ux66f2ux7ddaux9577ux3055ux9762ux7a4d}

与えられた曲線 {[} x=a(t-\sin t),\qquad y=a(1-\cos t). {]}

\subsection{\texorpdfstring{(1)
長さ(\(0\le t\le T\))}{(1) 長さ(0\textbackslash le t\textbackslash le T)}}\label{ux9577ux30550le-tle-t}

\subsubsection{解くための公式}\label{ux89e3ux304fux305fux3081ux306eux516cux5f0f-5}

媒介変数の弧長: {[}
L=\int\_\{0\}\^{}\{T\}\sqrt{\left(\frac{dx}{dt}\right)^2+\left(\frac{dy}{dt}\right)^2},dt.
{]}

\subsubsection{使う分野の知識}\label{ux4f7fux3046ux5206ux91ceux306eux77e5ux8b58-5}

\begin{itemize}
\tightlist
\item
  媒介変数表示、弧長公式
\item
  三角恒等式 \(\sin^2 t+\cos^2 t=1\)
\end{itemize}

\subsubsection{解くための考え方}\label{ux89e3ux304fux305fux3081ux306eux8003ux3048ux65b9-5}

\begin{itemize}
\tightlist
\item
  \(dx/dt,dy/dt\) を作ってルート内を整理(恒等式で簡約される形が多い)
\item
  整理後に1変数積分を実行
\end{itemize}

\begin{center}\rule{0.5\linewidth}{0.5pt}\end{center}

\subsection{\texorpdfstring{(2)
\(x\)軸と囲む面積(\(0\le t\le 2\pi\))}{(2) x軸と囲む面積(0\textbackslash le t\textbackslash le 2\textbackslash pi)}}\label{xux8ef8ux3068ux56f2ux3080ux9762ux7a4d0le-tle-2pi}

\subsubsection{解くための公式}\label{ux89e3ux304fux305fux3081ux306eux516cux5f0f-6}

{[} A=\int y,dx=\int\_\{0\}\^{}\{2\pi\}y(t)\frac{dx}{dt},dt. {]}

\subsubsection{使う分野の知識}\label{ux4f7fux3046ux5206ux91ceux306eux77e5ux8b58-6}

\begin{itemize}
\tightlist
\item
  線積分(面積公式)
\item
  パラメトリック積分
\end{itemize}

\subsubsection{解くための考え方}\label{ux89e3ux304fux305fux3081ux306eux8003ux3048ux65b9-6}

\begin{itemize}
\tightlist
\item
  (1)で得た \(dx/dt\) を流用
\item
  \(y(t)\,dx/dt\) を三角関数として整理し、周期性も使って積分
\end{itemize}

\begin{center}\rule{0.5\linewidth}{0.5pt}\end{center}

\section{問題3(極値と最大・最小)}\label{ux554fux984c3ux6975ux5024ux3068ux6700ux5927ux6700ux5c0f}

\subsection{\texorpdfstring{(1) \(z=-\dfrac94 x^3-\dfrac23 y^3+9xy\)
の極値}{(1) z=-\textbackslash dfrac94 x\^{}3-\textbackslash dfrac23 y\^{}3+9xy の極値}}\label{z-dfrac94-x3-dfrac23-y39xy-ux306eux6975ux5024}

\subsubsection{解くための公式}\label{ux89e3ux304fux305fux3081ux306eux516cux5f0f-7}

臨界点: {[} z\_x=0,\qquad z\_y=0. {]} 2階判定(2変数): {[}
D=z\_\{xx\}z\_\{yy\}-z\_\{xy\}\^{}2. {]} {[}
D\textgreater0,~z\_\{xx\}\textgreater0 \Rightarrow \text{極小},\qquad
D\textgreater0,~z\_\{xx\}\textless0 \Rightarrow \text{極大},\qquad
D\textless0 \Rightarrow \text{鞍点}. {]}

\subsubsection{使う分野の知識}\label{ux4f7fux3046ux5206ux91ceux306eux77e5ux8b58-7}

\begin{itemize}
\tightlist
\item
  多変数微分、偏微分、2階微分判定
\item
  非線形連立方程式の解法(ここでは \(x,y\) が3次の式になる)
\end{itemize}

\subsubsection{解くための考え方}\label{ux89e3ux304fux305fux3081ux306eux8003ux3048ux65b9-7}

\begin{itemize}
\tightlist
\item
  まず \(z_x,z_y\) を計算して臨界点を列挙
\item
  次に \(z_{xx},z_{yy},z_{xy}\) で分類
\item
  要求されるなら臨界点へ代入して \(z\) の値も明記
\end{itemize}

\begin{center}\rule{0.5\linewidth}{0.5pt}\end{center}

\subsection{(2)
極大/極小が「最大/最小」かどうか}\label{ux6975ux5927ux6975ux5c0fux304cux6700ux5927ux6700ux5c0fux304bux3069ux3046ux304b}

\subsubsection{解くための公式}\label{ux89e3ux304fux305fux3081ux306eux516cux5f0f-8}

\begin{itemize}
\tightlist
\item
  局所極値 \(\Rightarrow\) 全体最大/最小 とは限らない
\item
  無限遠の挙動は支配項(最高次数項)で概ね判定できる\\
  (例えば、ある方向で \(z\to +\infty\) なら最大値は存在しない、など)
\end{itemize}

\subsubsection{使う分野の知識}\label{ux4f7fux3046ux5206ux91ceux306eux77e5ux8b58-8}

\begin{itemize}
\tightlist
\item
  関数の発散(無限遠解析)、存在判定(最大値・最小値の有無)
\end{itemize}

\subsubsection{解くための考え方}\label{ux89e3ux304fux305fux3081ux306eux8003ux3048ux65b9-8}

\begin{itemize}
\tightlist
\item
  \(x\to\infty\) や \(y\to\infty\) など複数方向で \(z\)
  の符号・発散先を確認
\item
  「別方向で \(+\infty\) と \(-\infty\)
  の両方へ行く」なら最大も最小も成立しない、等を論理で詰める
\end{itemize}

\begin{center}\rule{0.5\linewidth}{0.5pt}\end{center}

\section{問題4(極座標で二重積分)}\label{ux554fux984c4ux6975ux5ea7ux6a19ux3067ux4e8cux91cdux7a4dux5206}

\subsubsection{解くための公式}\label{ux89e3ux304fux305fux3081ux306eux516cux5f0f-9}

極座標変換: {[}
x=r\cos\theta,\qquad y=r\sin\theta,\qquad dx,dy=r,dr,d\theta. {]} 領域
\(x^2+y^2\le 9\) は {[} 0\le r\le 3,\qquad 0\le \theta\le 2\pi. {]} また
\(x^2+y^2=r^2\)。

\subsubsection{使う分野の知識}\label{ux4f7fux3046ux5206ux91ceux306eux77e5ux8b58-9}

\begin{itemize}
\tightlist
\item
  極座標変換、ヤコビアン \(r\)
\item
  半径のみ依存する関数の積分(\(\theta\) が分離できる)
\end{itemize}

\subsubsection{解くための考え方}\label{ux89e3ux304fux305fux3081ux306eux8003ux3048ux65b9-9}

\begin{itemize}
\tightlist
\item
  被積分関数が \(x^2+y^2\)(つまり
  \(r^2\))のみに依存する形なので極座標が最短
\item
  \(\theta\) 積分は \(2\pi\) の因子として抜け、\(r\) の1変数積分に落ちる
\end{itemize}

\begin{center}\rule{0.5\linewidth}{0.5pt}\end{center}

\section{問題5(級数:収束・発散判定)}\label{ux554fux984c5ux7d1aux6570ux53ceux675fux767aux6563ux5224ux5b9a}

\subsection{\texorpdfstring{(1)
\(\displaystyle \sum_{n=1}^{\infty}\frac{n}{(2\sqrt{n}-1)^2(\sqrt{n}+1)^2}\)}{(1) \textbackslash displaystyle \textbackslash sum\_\{n=1\}\^{}\{\textbackslash infty\}\textbackslash frac\{n\}\{(2\textbackslash sqrt\{n\}-1)\^{}2(\textbackslash sqrt\{n\}+1)\^{}2\}}}\label{displaystyle-sum_n1inftyfracn2sqrtn-12sqrtn12}

\subsubsection{解くための公式}\label{ux89e3ux304fux305fux3081ux306eux516cux5f0f-10}

\begin{itemize}
\tightlist
\item
  比較判定(あるいは極限比較)
\item
  漸近評価:\(n\to\infty\) で {[}
  (2\sqrt{n}-1)\sim 2\sqrt{n},\qquad (\sqrt{n}+1)\sim \sqrt{n}. {]}
\item
  \(p\)-級数:\(\sum 1/n^p\) は \(p>1\) で収束、\(p\le 1\) で発散
\end{itemize}

\subsubsection{使う分野の知識}\label{ux4f7fux3046ux5206ux91ceux306eux77e5ux8b58-10}

\begin{itemize}
\tightlist
\item
  漸近(支配項)、比較判定、\(p\)-級数
\end{itemize}

\subsubsection{解くための考え方}\label{ux89e3ux304fux305fux3081ux306eux8003ux3048ux65b9-10}

\begin{itemize}
\tightlist
\item
  分母の次数だけ読む:概ね {[}
  (2\sqrt{n})\textsuperscript{2(\sqrt{n})}2=4n\^{}2 {]} なので一般項は
  \(\sim \dfrac{1}{4n}\) と読める
\item
  調和級数型(\(\sum 1/n\))との比較で結論を出す
\end{itemize}

\begin{center}\rule{0.5\linewidth}{0.5pt}\end{center}

\subsection{\texorpdfstring{(2)
\(\displaystyle \sum_{n=1}^{\infty}\frac{(\sqrt{2})^n}{n^2}\)}{(2) \textbackslash displaystyle \textbackslash sum\_\{n=1\}\^{}\{\textbackslash infty\}\textbackslash frac\{(\textbackslash sqrt\{2\})\^{}n\}\{n\^{}2\}}}\label{displaystyle-sum_n1inftyfracsqrt2nn2}

\subsubsection{解くための公式}\label{ux89e3ux304fux305fux3081ux306eux516cux5f0f-11}

比の判定法: {[} a\_n=\frac{(\sqrt{2})^n}{n^2},\qquad
\lim\_\{n\to\infty\}\frac{a_{n+1}}{a_n}=L,\quad
L\textgreater1\Rightarrow \text{発散}. {]}

\subsubsection{使う分野の知識}\label{ux4f7fux3046ux5206ux91ceux306eux77e5ux8b58-11}

\begin{itemize}
\tightlist
\item
  比の判定法
\item
  「指数増加は多項式減衰に勝つ」
\end{itemize}

\subsubsection{解くための考え方}\label{ux89e3ux304fux305fux3081ux306eux8003ux3048ux65b9-11}

\begin{itemize}
\tightlist
\item
  比を取って {[} \frac{a_{n+1}}{a_n}
  =\sqrt{2}\cdot \frac{n^2}{(n+1)^2}\to \sqrt{2}\textgreater1 {]}
  を示せば一撃
\end{itemize}

\begin{center}\rule{0.5\linewidth}{0.5pt}\end{center}

\section{\texorpdfstring{問題6(\(\log(1+x)\)
の整級数展開)}{問題6(\textbackslash log(1+x) の整級数展開)}}\label{ux554fux984c6log1x-ux306eux6574ux7d1aux6570ux5c55ux958b}

\subsection{\texorpdfstring{(1)
\(\dfrac{1}{1-x}=\sum_{n=0}^{\infty}x^n\)
を用いる}{(1) \textbackslash dfrac\{1\}\{1-x\}=\textbackslash sum\_\{n=0\}\^{}\{\textbackslash infty\}x\^{}n を用いる}}\label{dfrac11-xsum_n0inftyxn-ux3092ux7528ux3044ux308b}

\subsubsection{解くための公式}\label{ux89e3ux304fux305fux3081ux306eux516cux5f0f-12}

幾何級数(\(|x|<1\)): {[}
\frac{1}{1-x}=\sum\emph{\{n=0\}\textsuperscript{\{\infty\}x}n. {]} 置換
\(x\mapsto -x\) により {[}
\frac{1}{1+x}=\frac{1}{1-(-x)}=\sum}\{n=0\}\textsuperscript{\{\infty\}(-1)}n
x\^{}n\qquad(\textbar x\textbar\textless1). {]} 両辺を項別積分して {[}
\log(1+x)=\int \frac{1}{1+x},dx
=\sum\emph{\{n=0\}\textsuperscript{\{\infty\}(-1)}n\frac{x^{n+1}}{n+1}
=\sum}\{n=1\}\textsuperscript{\{\infty\}(-1)}\{n-1\}\frac{x^n}{n}. {]}
積分定数は \(\log(1+0)=0\) で決める。

\subsection{(2)
級数展開が成立する範囲}\label{ux7d1aux6570ux5c55ux958bux304cux6210ux7acbux3059ux308bux7bc4ux56f2}

\subsubsection{使う分野の知識}\label{ux4f7fux3046ux5206ux91ceux306eux77e5ux8b58-12}

\begin{itemize}
\tightlist
\item
  べき級数、項別積分、収束半径
\item
  端点での収束判定(必要なら交代級数判定など)
\end{itemize}

\subsubsection{解くための考え方}\label{ux89e3ux304fux305fux3081ux306eux8003ux3048ux65b9-12}

\begin{itemize}
\tightlist
\item
  まず変形で得た等式は \(|x|<1\) で保証される
\item
  「範囲を求めよ」がある場合は \(x=1,-1\)
  を別途判定して最終範囲を確定する
\end{itemize}




\end{document}
