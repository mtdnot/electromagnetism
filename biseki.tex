% Options for packages loaded elsewhere
\PassOptionsToPackage{unicode}{hyperref}
\PassOptionsToPackage{hyphens}{url}
\PassOptionsToPackage{dvipsnames,svgnames,x11names}{xcolor}
%
\documentclass[
]{article}

\usepackage{amsmath,amssymb}
\usepackage{iftex}
\ifPDFTeX
  \usepackage[T1]{fontenc}
  \usepackage[utf8]{inputenc}
  \usepackage{textcomp} % provide euro and other symbols
\else % if luatex or xetex
  \usepackage{unicode-math}
  \defaultfontfeatures{Scale=MatchLowercase}
  \defaultfontfeatures[\rmfamily]{Ligatures=TeX,Scale=1}
\fi
\usepackage{lmodern}
\ifPDFTeX\else  
    % xetex/luatex font selection
\fi
% Use upquote if available, for straight quotes in verbatim environments
\IfFileExists{upquote.sty}{\usepackage{upquote}}{}
\IfFileExists{microtype.sty}{% use microtype if available
  \usepackage[]{microtype}
  \UseMicrotypeSet[protrusion]{basicmath} % disable protrusion for tt fonts
}{}
\makeatletter
\@ifundefined{KOMAClassName}{% if non-KOMA class
  \IfFileExists{parskip.sty}{%
    \usepackage{parskip}
  }{% else
    \setlength{\parindent}{0pt}
    \setlength{\parskip}{6pt plus 2pt minus 1pt}}
}{% if KOMA class
  \KOMAoptions{parskip=half}}
\makeatother
\usepackage{xcolor}
\setlength{\emergencystretch}{3em} % prevent overfull lines
\setcounter{secnumdepth}{5}
% Make \paragraph and \subparagraph free-standing
\makeatletter
\ifx\paragraph\undefined\else
  \let\oldparagraph\paragraph
  \renewcommand{\paragraph}{
    \@ifstar
      \xxxParagraphStar
      \xxxParagraphNoStar
  }
  \newcommand{\xxxParagraphStar}[1]{\oldparagraph*{#1}\mbox{}}
  \newcommand{\xxxParagraphNoStar}[1]{\oldparagraph{#1}\mbox{}}
\fi
\ifx\subparagraph\undefined\else
  \let\oldsubparagraph\subparagraph
  \renewcommand{\subparagraph}{
    \@ifstar
      \xxxSubParagraphStar
      \xxxSubParagraphNoStar
  }
  \newcommand{\xxxSubParagraphStar}[1]{\oldsubparagraph*{#1}\mbox{}}
  \newcommand{\xxxSubParagraphNoStar}[1]{\oldsubparagraph{#1}\mbox{}}
\fi
\makeatother


\providecommand{\tightlist}{%
  \setlength{\itemsep}{0pt}\setlength{\parskip}{0pt}}\usepackage{longtable,booktabs,array}
\usepackage{calc} % for calculating minipage widths
% Correct order of tables after \paragraph or \subparagraph
\usepackage{etoolbox}
\makeatletter
\patchcmd\longtable{\par}{\if@noskipsec\mbox{}\fi\par}{}{}
\makeatother
% Allow footnotes in longtable head/foot
\IfFileExists{footnotehyper.sty}{\usepackage{footnotehyper}}{\usepackage{footnote}}
\makesavenoteenv{longtable}
\usepackage{graphicx}
\makeatletter
\newsavebox\pandoc@box
\newcommand*\pandocbounded[1]{% scales image to fit in text height/width
  \sbox\pandoc@box{#1}%
  \Gscale@div\@tempa{\textheight}{\dimexpr\ht\pandoc@box+\dp\pandoc@box\relax}%
  \Gscale@div\@tempb{\linewidth}{\wd\pandoc@box}%
  \ifdim\@tempb\p@<\@tempa\p@\let\@tempa\@tempb\fi% select the smaller of both
  \ifdim\@tempa\p@<\p@\scalebox{\@tempa}{\usebox\pandoc@box}%
  \else\usebox{\pandoc@box}%
  \fi%
}
% Set default figure placement to htbp
\def\fps@figure{htbp}
\makeatother

\makeatletter
\@ifpackageloaded{caption}{}{\usepackage{caption}}
\AtBeginDocument{%
\ifdefined\contentsname
  \renewcommand*\contentsname{Table of contents}
\else
  \newcommand\contentsname{Table of contents}
\fi
\ifdefined\listfigurename
  \renewcommand*\listfigurename{List of Figures}
\else
  \newcommand\listfigurename{List of Figures}
\fi
\ifdefined\listtablename
  \renewcommand*\listtablename{List of Tables}
\else
  \newcommand\listtablename{List of Tables}
\fi
\ifdefined\figurename
  \renewcommand*\figurename{Figure}
\else
  \newcommand\figurename{Figure}
\fi
\ifdefined\tablename
  \renewcommand*\tablename{Table}
\else
  \newcommand\tablename{Table}
\fi
}
\@ifpackageloaded{float}{}{\usepackage{float}}
\floatstyle{ruled}
\@ifundefined{c@chapter}{\newfloat{codelisting}{h}{lop}}{\newfloat{codelisting}{h}{lop}[chapter]}
\floatname{codelisting}{Listing}
\newcommand*\listoflistings{\listof{codelisting}{List of Listings}}
\makeatother
\makeatletter
\makeatother
\makeatletter
\@ifpackageloaded{caption}{}{\usepackage{caption}}
\@ifpackageloaded{subcaption}{}{\usepackage{subcaption}}
\makeatother

\usepackage{bookmark}

\IfFileExists{xurl.sty}{\usepackage{xurl}}{} % add URL line breaks if available
\urlstyle{same} % disable monospaced font for URLs
\hypersetup{
  pdftitle={過去問分析(Exam2024\_prb)},
  colorlinks=true,
  linkcolor={blue},
  filecolor={Maroon},
  citecolor={Blue},
  urlcolor={Blue},
  pdfcreator={LaTeX via pandoc}}


\title{過去問分析(Exam2024\_prb)}
\author{}
\date{}

\begin{document}
\maketitle

\renewcommand*\contentsname{Table of contents}
{
\hypersetup{linkcolor=}
\setcounter{tocdepth}{3}
\tableofcontents
}

\section{対象ファイル}\label{ux5bfeux8c61ux30d5ux30a1ux30a4ux30eb}

Exam2024\_prb(問題1〜6)

\begin{center}\rule{0.5\linewidth}{0.5pt}\end{center}

\section{問題1(空欄補充:微分・極値・重積分・変数変換)}\label{ux554fux984c1ux7a7aux6b04ux88dcux5145ux5faeux5206ux6975ux5024ux91cdux7a4dux5206ux5909ux6570ux5909ux63db}

\subsection{\texorpdfstring{(1)
2階偏導関数:\(z=\log|x^2-2y|\)}{(1) 2階偏導関数:z=\textbackslash log\textbar x\^{}2-2y\textbar{}}}\label{ux968eux504fux5c0eux95a2ux6570zlogx2-2y}

\subsubsection{解くための公式}\label{ux89e3ux304fux305fux3081ux306eux516cux5f0f}

\begin{itemize}
\tightlist
\item
  置換 \(u=x^2-2y\) とすると {[} z=\log\textbar u\textbar,\quad 
  \frac{\partial z}{\partial x}=\frac{1}{u}\frac{\partial u}{\partial x},\quad
  \frac{\partial z}{\partial y}=\frac{1}{u}\frac{\partial u}{\partial y}\quad (u\neq 0)
  {]}
\item
  2階偏導・混合偏導 {[}
  \frac{\partial^2 z}{\partial x^2}=\frac{\partial}{\partial x}\left(\frac{\partial z}{\partial x}\right),
  \quad
  \frac{\partial^2 z}{\partial y\,\partial x}=\frac{\partial}{\partial y}\left(\frac{\partial z}{\partial x}\right)
  {]}
\end{itemize}

\subsubsection{使う分野の知識}\label{ux4f7fux3046ux5206ux91ceux306eux77e5ux8b58}

\begin{itemize}
\tightlist
\item
  多変数微分、合成関数の微分(連鎖律)
\item
  \(\log|u|\) の微分は \(\log u\) と同形だが、\textbf{定義域として
  \(u\neq 0\) を意識}
\end{itemize}

\subsubsection{解くための考え方}\label{ux89e3ux304fux305fux3081ux306eux8003ux3048ux65b9}

\begin{itemize}
\tightlist
\item
  まず \(u=x^2-2y\)
  と置き、\(\partial z/\partial x,\partial z/\partial y\) を出す
\item
  その式をもう一度微分して2階へ(分数の微分で符号ミス注意)
\end{itemize}

\begin{center}\rule{0.5\linewidth}{0.5pt}\end{center}

\subsection{\texorpdfstring{(2)
連鎖律:\(z=f(x,y),\ x=a\cos t,\ y=b\sin t\)}{(2) 連鎖律:z=f(x,y),\textbackslash{} x=a\textbackslash cos t,\textbackslash{} y=b\textbackslash sin t}}\label{ux9023ux9396ux5f8bzfxy-xacos-t-ybsin-t}

\subsubsection{解くための公式}\label{ux89e3ux304fux305fux3081ux306eux516cux5f0f-1}

\begin{itemize}
\tightlist
\item
  1階: {[} \frac{dz}{dt}=f\_x\frac{dx}{dt}+f\_y\frac{dy}{dt} {]}
\item
  2階(標準形): {[} \frac{d^2z}{dt^2}
  =f\_\{xx\}\left(\frac{dx}{dt}\right)\^{}2
  +2f\_\{xy\}\left(\frac{dx}{dt}\right)\left(\frac{dy}{dt}\right)
  +f\_\{yy\}\left(\frac{dy}{dt}\right)\^{}2 +f\_x\frac{d^2x}{dt^2}
  +f\_y\frac{d^2y}{dt^2} {]}
\end{itemize}

\subsubsection{使う分野の知識}\label{ux4f7fux3046ux5206ux91ceux306eux77e5ux8b58-1}

\begin{itemize}
\tightlist
\item
  多変数の連鎖律(1階・2階)、偏微分記号
\end{itemize}

\subsubsection{解くための考え方}\label{ux89e3ux304fux305fux3081ux306eux8003ux3048ux65b9-1}

\begin{itemize}
\tightlist
\item
  先に \(dx/dt,dy/dt,d^2x/dt^2,d^2y/dt^2\) を確定(\(\sin,\cos\)
  の微分は機械的)
\item
  2階は「\(f\) の2階偏微分ブロック」+「\(f\)
  の1階偏微分×2階導関数ブロック」に分けて整理
\end{itemize}

\begin{center}\rule{0.5\linewidth}{0.5pt}\end{center}

\subsection{\texorpdfstring{(3) 2次関数の最大値:\(f,g,h\)
の比較}{(3) 2次関数の最大値:f,g,h の比較}}\label{ux6b21ux95a2ux6570ux306eux6700ux5927ux5024fgh-ux306eux6bd4ux8f03}

\subsubsection{解くための公式}\label{ux89e3ux304fux305fux3081ux306eux516cux5f0f-2}

\begin{itemize}
\tightlist
\item
  極値点:\(\nabla Q=0\)
\item
  判定(ヘッセ行列 \(H\)): {[} H=

  \begin{pmatrix}Q_{xx}&Q_{xy}\\Q_{yx}&Q_{yy}\end{pmatrix}

  ,\quad D=\det H=Q\_\{xx\}Q\_\{yy\}-Q\_\{xy\}\^{}2 {]} {[}
  D\textgreater0,~Q\_\{xx\}\textless0 \Rightarrow \text{極大(負定値)}
  \quad D\textgreater0,~Q\_\{xx\}\textgreater0
  \Rightarrow \text{極小(正定値)} \quad D\textless0
  \Rightarrow \text{鞍点(極値なし)} {]}
\item
  2次関数が負定値なら \textbf{全体最大値}
  をもつ(上に凸ではなく「下に凸の逆」)
\end{itemize}

\subsubsection{使う分野の知識}\label{ux4f7fux3046ux5206ux91ceux306eux77e5ux8b58-2}

\begin{itemize}
\tightlist
\item
  多変数関数の極値、二次形式、ヘッセ行列の符号判定
\end{itemize}

\subsubsection{解くための考え方}\label{ux89e3ux304fux305fux3081ux306eux8003ux3048ux65b9-2}

\begin{itemize}
\tightlist
\item
  まず各関数の \(H\) を見て「最大値を持ちうる(負定値か)」を即判定
\item
  候補だけ \(\nabla Q=0\) を解いて点を出し、代入して最大値を比較
\end{itemize}

\begin{center}\rule{0.5\linewidth}{0.5pt}\end{center}

\subsection{\texorpdfstring{(4) 二重積分:\(\iint_D x\,dxdy\)(\(y\)
の上下が2つの放物線)}{(4) 二重積分:\textbackslash iint\_D x\textbackslash,dxdy(y の上下が2つの放物線)}}\label{ux4e8cux91cdux7a4dux5206iint_d-xdxdyy-ux306eux4e0aux4e0bux304c2ux3064ux306eux653eux7269ux7dda}

\subsubsection{解くための公式}\label{ux89e3ux304fux305fux3081ux306eux516cux5f0f-3}

\begin{itemize}
\tightlist
\item
  縦に積分: {[} \iint\emph{D
  x,dxdy=\int}\{\alpha\}\^{}\{\beta\}\left(\int\_\{y=\mathrm{lower}(x)\}\^{}\{\mathrm{upper}(x)\}
  x,dy\right)dx {]}
\item
  内側は {[} \int x,dy=x\bigl(\mathrm{upper}(x)-\mathrm{lower}(x)\bigr)
  {]}
\item
  \(x\) 範囲は \(\mathrm{lower}(x)\le \mathrm{upper}(x)\) から決める
\end{itemize}

\subsubsection{使う分野の知識}\label{ux4f7fux3046ux5206ux91ceux306eux77e5ux8b58-3}

\begin{itemize}
\tightlist
\item
  二重積分、領域分割、曲線の交点(2次方程式)
\end{itemize}

\subsubsection{解くための考え方}\label{ux89e3ux304fux305fux3081ux306eux8003ux3048ux65b9-3}

\begin{itemize}
\tightlist
\item
  交点で \(\alpha,\beta\) を確定し、内側積分は「長さ×\(x\)」に落とす
\item
  外側は1変数積分として計算
\end{itemize}

\begin{center}\rule{0.5\linewidth}{0.5pt}\end{center}

\subsection{\texorpdfstring{(5) 変数変換:\(u=x+y,\ v=3x-y\) による
\(\iint_D y\,dxdy\)}{(5) 変数変換:u=x+y,\textbackslash{} v=3x-y による \textbackslash iint\_D y\textbackslash,dxdy}}\label{ux5909ux6570ux5909ux63dbuxy-v3x-y-ux306bux3088ux308b-iint_d-ydxdy}

\subsubsection{解くための公式}\label{ux89e3ux304fux305fux3081ux306eux516cux5f0f-4}

\begin{itemize}
\tightlist
\item
  線形変換: {[} u=x+y,\quad v=3x-y {]}
\item
  ヤコビアン: {[}
  dxdy=\left\textbar{}\frac{\partial(x,y)}{\partial(u,v)}\right\textbar,dudv
  =\frac{1}{\left|\frac{\partial(u,v)}{\partial(x,y)}\right|},dudv {]}
\item
  被積分関数を \(y=y(u,v)\) として書き換える(\(x,y\) を連立一次で解く)
\end{itemize}

\subsubsection{使う分野の知識}\label{ux4f7fux3046ux5206ux91ceux306eux77e5ux8b58-4}

\begin{itemize}
\tightlist
\item
  2変数の変数変換、ヤコビアン、領域の写像(不等式が長方形になる典型)
\end{itemize}

\subsubsection{解くための考え方}\label{ux89e3ux304fux305fux3081ux306eux8003ux3048ux65b9-4}

\begin{itemize}
\tightlist
\item
  まず新領域 \(K\) は条件から即決(\(0\le u\le 2,\ 0\le v\le 3\) の形)
\item
  次に逆変換で \(y(u,v)\) と \(|J|\) を出して、長方形上の通常積分にする
\end{itemize}

\begin{center}\rule{0.5\linewidth}{0.5pt}\end{center}

\section{問題2(媒介変数曲線:長さ・面積)}\label{ux554fux984c2ux5a92ux4ecbux5909ux6570ux66f2ux7ddaux9577ux3055ux9762ux7a4d}

\subsection{\texorpdfstring{(1)
長さ(\(0\le t\le T\))}{(1) 長さ(0\textbackslash le t\textbackslash le T)}}\label{ux9577ux30550le-tle-t}

\subsubsection{解くための公式}\label{ux89e3ux304fux305fux3081ux306eux516cux5f0f-5}

{[}
L=\int\_0\^{}T\sqrt{\left(\frac{dx}{dt}\right)^2+\left(\frac{dy}{dt}\right)^2},dt
{]}

\subsubsection{使う分野の知識}\label{ux4f7fux3046ux5206ux91ceux306eux77e5ux8b58-5}

\begin{itemize}
\tightlist
\item
  媒介変数表示、弧長公式、三角恒等式 \(\sin^2 t+\cos^2 t=1\)
\end{itemize}

\subsubsection{解くための考え方}\label{ux89e3ux304fux305fux3081ux306eux8003ux3048ux65b9-5}

\begin{itemize}
\tightlist
\item
  \(dx/dt,dy/dt\) を計算してルート内を恒等式で簡約し、積分する
\end{itemize}

\begin{center}\rule{0.5\linewidth}{0.5pt}\end{center}

\subsection{\texorpdfstring{(2)
面積(\(0\le t\le 2\pi\))}{(2) 面積(0\textbackslash le t\textbackslash le 2\textbackslash pi)}}\label{ux9762ux7a4d0le-tle-2pi}

\subsubsection{解くための公式}\label{ux89e3ux304fux305fux3081ux306eux516cux5f0f-6}

{[} A=\int y,dx=\int\_0\^{}\{2\pi\}y(t)\frac{dx}{dt},dt {]}

\subsubsection{使う分野の知識}\label{ux4f7fux3046ux5206ux91ceux306eux77e5ux8b58-6}

\begin{itemize}
\tightlist
\item
  線積分による面積、パラメトリック積分
\end{itemize}

\subsubsection{解くための考え方}\label{ux89e3ux304fux305fux3081ux306eux8003ux3048ux65b9-6}

\begin{itemize}
\tightlist
\item
  (1)で得た \(dx/dt\) を流用して \(y(t)\,dx/dt\)
  を整理し、周期性も使って積分
\end{itemize}

\begin{center}\rule{0.5\linewidth}{0.5pt}\end{center}

\section{問題3(極値と最大最小)}\label{ux554fux984c3ux6975ux5024ux3068ux6700ux5927ux6700ux5c0f}

\subsection{\texorpdfstring{(1)
極値調査:\(z=-\frac94 x^3-\frac23 y^3+9xy\)}{(1) 極値調査:z=-\textbackslash frac94 x\^{}3-\textbackslash frac23 y\^{}3+9xy}}\label{ux6975ux5024ux8abfux67fbz-frac94-x3-frac23-y39xy}

\subsubsection{解くための公式}\label{ux89e3ux304fux305fux3081ux306eux516cux5f0f-7}

\begin{itemize}
\tightlist
\item
  臨界点: {[} z\_x=0,\quad z\_y=0 {]}
\item
  判定: {[} D=z\_\{xx\}z\_\{yy\}-z\_\{xy\}\^{}2 {]} {[}
  D\textgreater0,~z\_\{xx\}\textgreater0 \Rightarrow \text{極小},\quad
  D\textgreater0,~z\_\{xx\}\textless0 \Rightarrow \text{極大},\quad
  D\textless0 \Rightarrow \text{鞍点} {]}
\end{itemize}

\subsubsection{使う分野の知識}\label{ux4f7fux3046ux5206ux91ceux306eux77e5ux8b58-7}

\begin{itemize}
\tightlist
\item
  多変数微分、2階微分判定、非線形連立方程式
\end{itemize}

\subsubsection{解くための考え方}\label{ux89e3ux304fux305fux3081ux306eux8003ux3048ux65b9-7}

\begin{itemize}
\tightlist
\item
  まず臨界点を列挙→ヘッセ情報で分類→最後に \(z\) の値を代入で明記
\end{itemize}

\begin{center}\rule{0.5\linewidth}{0.5pt}\end{center}

\subsection{(2)
極大/極小が最大/最小か}\label{ux6975ux5927ux6975ux5c0fux304cux6700ux5927ux6700ux5c0fux304b}

\subsubsection{解くための公式}\label{ux89e3ux304fux305fux3081ux306eux516cux5f0f-8}

\begin{itemize}
\tightlist
\item
  局所極値 \(\not\Rightarrow\) 全体最大/最小
\item
  無限遠での挙動:支配項(最高次数項)で \(z\to \pm\infty\) を判定
\end{itemize}

\subsubsection{使う分野の知識}\label{ux4f7fux3046ux5206ux91ceux306eux77e5ux8b58-8}

\begin{itemize}
\tightlist
\item
  最適化の存在判定、無限遠解析(方向による発散)
\end{itemize}

\subsubsection{解くための考え方}\label{ux89e3ux304fux305fux3081ux306eux8003ux3048ux65b9-8}

\begin{itemize}
\tightlist
\item
  \(x\) や \(y\) を大きくしたときに \(z\)
  がどちらへ発散するか、複数方向で確認
\item
  一方で \(+\infty\)、別方向で \(-\infty\)
  へ行くなら最大/最小は存在しない、などを論理で示す
\end{itemize}

\begin{center}\rule{0.5\linewidth}{0.5pt}\end{center}

\section{問題4(極座標で二重積分)}\label{ux554fux984c4ux6975ux5ea7ux6a19ux3067ux4e8cux91cdux7a4dux5206}

\subsubsection{解くための公式}\label{ux89e3ux304fux305fux3081ux306eux516cux5f0f-9}

\begin{itemize}
\tightlist
\item
  極座標: {[} x=r\cos\theta,\quad y=r\sin\theta,\quad dxdy=r,dr,d\theta
  {]}
\item
  領域 \(x^2+y^2\le 9\): {[} 0\le r\le 3,\quad 0\le \theta\le 2\pi {]}
\end{itemize}

\subsubsection{使う分野の知識}\label{ux4f7fux3046ux5206ux91ceux306eux77e5ux8b58-9}

\begin{itemize}
\tightlist
\item
  極座標変換、ヤコビアン \(r\)
\end{itemize}

\subsubsection{解くための考え方}\label{ux89e3ux304fux305fux3081ux306eux8003ux3048ux65b9-9}

\begin{itemize}
\tightlist
\item
  被積分関数が \(x^2+y^2=r^2\) のみ依存なので、\(\theta\) は \(2\pi\)
  で因数分離できる
\end{itemize}

\begin{center}\rule{0.5\linewidth}{0.5pt}\end{center}

\section{問題5(級数の収束・発散)}\label{ux554fux984c5ux7d1aux6570ux306eux53ceux675fux767aux6563}

\subsection{\texorpdfstring{(1)
\(\sum_{n=1}^\infty \dfrac{n}{(2\sqrt{n}-1)^2(\sqrt{n}+1)^2}\)}{(1) \textbackslash sum\_\{n=1\}\^{}\textbackslash infty \textbackslash dfrac\{n\}\{(2\textbackslash sqrt\{n\}-1)\^{}2(\textbackslash sqrt\{n\}+1)\^{}2\}}}\label{sum_n1infty-dfracn2sqrtn-12sqrtn12}

\subsubsection{解くための公式}\label{ux89e3ux304fux305fux3081ux306eux516cux5f0f-10}

\begin{itemize}
\tightlist
\item
  比較判定/極限比較
\item
  漸近:\((2\sqrt{n}-1)\sim 2\sqrt{n},\ (\sqrt{n}+1)\sim \sqrt{n}\)(\(n\to\infty\))
\end{itemize}

\subsubsection{使う分野の知識}\label{ux4f7fux3046ux5206ux91ceux306eux77e5ux8b58-10}

\begin{itemize}
\tightlist
\item
  漸近評価、\(p\)-級数 \(\sum 1/n^p\)(\(p>1\) で収束)
\end{itemize}

\subsubsection{解くための考え方}\label{ux89e3ux304fux305fux3081ux306eux8003ux3048ux65b9-10}

\begin{itemize}
\tightlist
\item
  分母の次数を読む:概ね \((2\sqrt{n})^2(\sqrt{n})^2=4n^2\)
\item
  よって一般項は \(n/(4n^2)\sim 1/(4n)\) の調和級数型と比較する
\end{itemize}

\begin{center}\rule{0.5\linewidth}{0.5pt}\end{center}

\subsection{\texorpdfstring{(2)
\(\sum_{n=1}^\infty \dfrac{(\sqrt{2})^n}{n^2}\)}{(2) \textbackslash sum\_\{n=1\}\^{}\textbackslash infty \textbackslash dfrac\{(\textbackslash sqrt\{2\})\^{}n\}\{n\^{}2\}}}\label{sum_n1infty-dfracsqrt2nn2}

\subsubsection{解くための公式}\label{ux89e3ux304fux305fux3081ux306eux516cux5f0f-11}

\begin{itemize}
\tightlist
\item
  比の判定法: {[}
  \lim\_\{n\to\infty\}\frac{a_{n+1}}{a_n}=L,\quad L\textgreater1
  \Rightarrow \text{発散} {]}
\end{itemize}

\subsubsection{使う分野の知識}\label{ux4f7fux3046ux5206ux91ceux306eux77e5ux8b58-11}

\begin{itemize}
\tightlist
\item
  指数増加は多項式減衰に勝つ、比の判定法
\end{itemize}

\subsubsection{解くための考え方}\label{ux89e3ux304fux305fux3081ux306eux8003ux3048ux65b9-11}

\begin{itemize}
\tightlist
\item
  比を取ると {[}
  \frac{a_{n+1}}{a_n}=\sqrt{2}\cdot\frac{n^2}{(n+1)^2}\to \sqrt{2}\textgreater1
  {]} で一撃
\end{itemize}

\begin{center}\rule{0.5\linewidth}{0.5pt}\end{center}

\section{\texorpdfstring{問題6(\(\log(1+x)\)
の整級数)}{問題6(\textbackslash log(1+x) の整級数)}}\label{ux554fux984c6log1x-ux306eux6574ux7d1aux6570}

\subsubsection{解くための公式}\label{ux89e3ux304fux305fux3081ux306eux516cux5f0f-12}

\begin{itemize}
\tightlist
\item
  幾何級数(\(|x|<1\)): {[}
  \frac{1}{1-x}=\sum\_\{n=0\}\textsuperscript{\{\infty\}x}n {]}
\item
  置換 \(x\mapsto -x\): {[}
  \frac{1}{1+x}=\frac{1}{1-(-x)}=\sum\_\{n=0\}\textsuperscript{\{\infty\}(-1)}n
  x\^{}n\quad (\textbar x\textbar\textless1) {]}
\item
  項別積分: {[} \log(1+x)=\int \frac{1}{1+x},dx
  =\sum\emph{\{n=0\}\textsuperscript{\{\infty\}(-1)}n\frac{x^{n+1}}{n+1}
  =\sum}\{n=1\}\textsuperscript{\{\infty\}(-1)}\{n-1\}\frac{x^{n}}{n}
  {]} かつ \(\log(1+0)=0\) で積分定数を決める
\end{itemize}

\subsubsection{使う分野の知識}\label{ux4f7fux3046ux5206ux91ceux306eux77e5ux8b58-12}

\begin{itemize}
\tightlist
\item
  べき級数、項別積分、収束半径(まず \(|x|<1\))
\item
  端点 \(x=\pm 1\) は別判定(問題が範囲も要求しているため)
\end{itemize}

\subsubsection{解くための考え方}\label{ux89e3ux304fux305fux3081ux306eux8003ux3048ux65b9-12}

\begin{itemize}
\tightlist
\item
  「与えられた級数 \(\to\) 形を合わせる(\(-x\))」→「積分して \(\log\)
  を作る」→「定数を初期条件で決める」
\item
  成立範囲は基本 \(|x|<1\)、端点は収束判定で詰める
\end{itemize}

\begin{center}\rule{0.5\linewidth}{0.5pt}\end{center}




\end{document}
